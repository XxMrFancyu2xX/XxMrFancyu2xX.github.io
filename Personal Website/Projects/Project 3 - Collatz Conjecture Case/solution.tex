The Collatz conjecture is an unsolved problem in mathematics that involves a sequence defined as follows: start with any positive integer \(n\). Then each term is obtained from the previous term as follows:
\begin{itemize}
    \item If the previous term is even, the next term is one half of the previous term.
    \item If the previous term is odd, the next term is three times the previous term plus one.
\end{itemize}
The conjecture states that no matter what value of \(n\) you start with, the sequence will always eventually reach 1.

We define the set \(N_k\) as follows:
\[
N_k = \{ n \in \mathbb{N} \mid n \geq k, k \in \mathbb{N} \}
\]

Suppose we have a number of the form \(2^n + 1\), where \(n \in N_1\). We can prove by induction that this number is odd for all such values.

For \(n = 1\):
\[
2^1 + 1 = 2 + 1 = 3
\]
3 is odd, so the base case holds.

Assume for some \(k \geq 1\), the number \(2^k + 1\) is odd. We need to show that \(2^{k+1} + 1\) is also odd.
\[
2^{k+1} + 1 = 2 \cdot 2^k + 1 = 2m + 1
\]
Since \(2m\) is even, \(2m + 1\) is odd. Thus, \(2^{k+1} + 1\) is odd.

By mathematical induction, \(2^n + 1\) is odd for all \(n \in N_1\).

We denote:
\begin{align*}
C_0 &= 2^n + 1 \\
C_1 &= 3 \cdot (2^n + 1) + 1 = 3 \cdot 2^n + 4
\end{align*}

\begin{itemize}
    \item \(C_1 = 3^1 \cdot 2^{n-0} + 4\) is even for \(N_1\)
    \item \(C_2 = 3^1 \cdot 2^{n-1} + 2\) is even for \(N_2\)
    \item \(C_3 = 3^1 \cdot 2^{n-2} + 1\) is odd for \(N_3\)
    \item \(C_4 = 3^2 \cdot 2^{n-2} + 4\) is even for \(N_3\)
    \item \(C_5 = 3^2 \cdot 2^{n-3} + 2\) is even for \(N_4\)
    \item \(C_6 = 3^2 \cdot 2^{n-4} + 1\) is odd for \(N_5\)
    \item \(C_7 = 3^3 \cdot 2^{n-4} + 4\) is even for \(N_5\)
\end{itemize}

\[
C_k = 3^{a(k)} \cdot 2^{n - b(k)} + c(k)
\]
where:
\begin{align*}
a(k) &= \left\lfloor \frac{k + 2}{3} \right\rfloor \\
b(k) &= \left\lfloor \frac{2k}{3} \right\rfloor \\
c(k) &= 4^k \mod 7
\end{align*}

\subsection{Behavior of \(C_k\)}
As \(k \to \infty\):
\[
C_k \approx c(k)
\]
where \(c(k)\) cycles through \(1, 4, 2\).

We have demonstrated that any number of the form \(2^n + 1\) will eventually reach and enter the repeating cycle \(1, 4, 2\) under the Collatz conjecture rules. This supports the idea that the conjecture holds true for this subset of integers.